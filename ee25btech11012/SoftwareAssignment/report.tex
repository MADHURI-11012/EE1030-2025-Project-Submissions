\let\negmedspace\undefined
\let\negthickspace\undefined
\documentclass[journal]{IEEEtran}
\usepackage[a5paper, margin=10mm, onecolumn]{geometry}
%\usepackage{lmodern} % Ensure lmodern is loaded for pdflatex
\usepackage{tfrupee} % Include tfrupee package

\setlength{\headheight}{1cm} % Set the height of the header box
\setlength{\headsep}{0mm}     % Set the distance between the header box and the top of the text

\usepackage{cite}
\usepackage{amsmath,amssymb,amsfonts,amsthm}
\usepackage{algorithmic}
\usepackage{graphicx}
\usepackage{textcomp}
\usepackage{xcolor}
\usepackage{txfonts}
\usepackage{listings}
\usepackage{enumitem}
\usepackage{mathtools}
\usepackage{gensymb}
\usepackage{comment}
\usepackage[breaklinks=true]{hyperref}
\usepackage{tkz-euclide} 
\usepackage{listings}
% \usepackage{gvv}                                        
\def\inputGnumericTable{}                                 
\usepackage[latin1]{inputenc}                                
\usepackage{color}                                            
\usepackage{array}                                            
\usepackage{longtable}                                       
\usepackage{calc}                                             
\usepackage{multirow}                                         
\usepackage{hhline}                                           
\usepackage{ifthen}                                           
\usepackage{lscape}
\usepackage{multicol}
\begin{document}

\bibliographystyle{IEEEtran}
\vspace{3cm}

\title{Software Project}
\author{EE25BTECH11012-BEERAM MADHURI}
% \maketitle
% \newpage
% \bigskip
{\let\newpage\relax\maketitle}

\renewcommand{\thefigure}{\theenumi}
\renewcommand{\thetable}{\theenumi}
\setlength{\intextsep}{10pt} % Space between text and floats


\numberwithin{equation}{enumi}
\numberwithin{figure}{enumi}
\renewcommand{\thetable}{\theenumi}

\section*{\textbf{Computing SVD of matrix A}}
\section*{\textbf{Summary of Strang's video}}
\textbf{Goal:}
$A = U \Sigma V^\top$ \\
Express matrix $A$ as product of $U$, $\Sigma$, $V^\top$ matrices \\
Where, 
\begin{align*}
U = Orthogonal matrix \\
V = Orthogonal matrix \\
\Sigma= diagonal matrix \\[1em]
\Sigma = \begin{bmatrix}
\sigma_1 & & \\
& \sigma_2 & \\
& & \ddots
\end{bmatrix} \\
\end{align*}
\textbf{Computation:}\\
let $u_1, u_2 \dots u_m$ be the unit vectors in the column space of $A$\\
$v_1, v_2 \dots v_n$ are the unit vectors in row space of $A$ \\[1em]
then, we are taking the typical vectors
\begin{align*}
\sigma_1 u_1 = A v_1 \\
\sigma_2 u_2 = A v_2 \\
\vdots \\
\sigma_r u_r = A v_r
\end{align*}
$u_1, u_2 \dots u_r$ are orthogonal to each other. \\
$v_1, v_2 \dots v_r$ are orthogonal to each other. 
\begin{align*}
A \begin{bmatrix} v_1 & v_2 & v_3 & \dots & v_r \end{bmatrix} = \begin{bmatrix} u_1 & u_2 & \dots & u_r \end{bmatrix}
\begin{bmatrix}
\sigma_1 & 0 & 0 & \dots \\
0 & \sigma_2 & & \dots \\
\vdots & & \ddots & \\
\end{bmatrix}
\end{align*}
$\therefore$ We have to find the orthonormal basis in the row space and  orthonormal basis in the column space of $A$ such that \\
\begin{align*}
    AV = U \Sigma
\end{align*}
as V and U are Orthogonal Matrices
\begin{align*}
    UU^\top=I\\
    VV^\top=I\\
    AVV^\top=U\Sigma V^\top\\
    A=U\Sigma V^\top
\end{align*}
Multiplying $A^\top$ with A
\begin{align*}
    A^\top A=(V\Sigma ^\top U^\top)(U\Sigma V^\top)\\
   A^\top A=V\Sigma ^\top (U^\top U)\Sigma V^\top\\
A^\top A=V\Sigma^\top \Sigma V^\top\\
    A^\top A=V \begin{pmatrix}
        \sigma_1 ^2 & 0&0 \cdots\\
        0&\sigma_2 ^2 & 0\cdots\\
        0&0&\sigma_2 ^2 \cdots\\
        \vdots & & \ddots & \\
    \end{pmatrix}V^\top
\end{align*}
let $A^\top$A =B and $\Sigma^\top \Sigma=M$\\
$B=VMV^\top$\\
In the same way 
\begin{align*}
    AA^\top = U\begin{pmatrix}
        \sigma_1 ^2 & 0&0 \cdots\\
        0&\sigma_2 ^2 & 0\cdots\\
        0&0&\sigma_2 ^2 \cdots\\
        \vdots & & \ddots & \\
    \end{pmatrix}U^\top
\end{align*}
$\therefore$ V is the Eigen vector matrix of matrix $A^\top$A\\
$\therefore$ $\Sigma$ is the diagonal matrix with eigen values of $A^\top A$ as it's diagonal entries. \\
$\therefore$ U is the Eigen vector matrix of matrix $AA^\top$\\
here,\\
$V_1, V_2, \cdots V_r$ are the orthonormal basis for rowspace of $A$.\\
$U_1, U_2, \cdots U_r$ are the orthonormal basis for columnspace of $A$.\\
$V_(r+1), \cdots V_n$ are the orthonormal basis for nullspace of $A$.\\
$U_(r+1),\cdots U_m$ are the orthonormal basis for nullspace of $A^\top$.\\
\section*{\textbf{Explaination of Implemented Algorithm}}
Power iteration with matrix deflation is done by svd.c code.\\
\begin{enumerate}
    \item The inner loop that calculates $v_new = matrix_multiply(A_T A v)$ is the power iteration. It's a method to find the dominant eigenvector (the $v_1$ corresponding to the largest singular value) of the matrix $A^\top A$.\\
 \item After finding the top singular triplet $(\sigma_1, u_1, v_1)$, we do two things:\\
 Reconstruct: We add this rank-1 matrix ($A_1 = \sigma_1 u_1 v_1^T$) to our final image, $A_k$.\\
 Deflate: We subtract that same rank-1 matrix from our temporary matrix ($A_temp = A_temp - A_1$).\\
  \item The main for loop repeats this $k$ times. On the second iteration, the power method automatically finds the next largest singular value/vector triplet because we've already "deflated" (removed) the first one.
 \end{enumerate}
\section*{\textbf{Why Power Iteration Algorithm over other algorithms}}
I studied about the following algorithms:
\begin{enumerate}
    \item Jacobi SVD Algorithm
    \item Golub-Kahan Algorithm
    \item Power Iteration
\end{enumerate}
\textbf{1. Jacobi SVD Algorithm:}\\
This algorithm applies a series of rotations to diagonalize the matrix\\
in this first we compute $A^\top A$\\
then apply Jacobi rotation :
\begin{align*}
    G^\top (A^\top A)G=diagonal\\
    \text{where, } G=\begin{pmatrix}
        \cos\theta & -\sin\theta\\
        \sin \theta & \cos\theta
    \end{pmatrix}
\end{align*}
equate non diagonal elements to zero to find the value of theta.\\
then the root of diagonal elements of this matrix gives $\sigma_1, \sigma_2$ etc.
\begin{align*}
    V=G\\
    U=AV\Sigma^-1
\end{align*}
It computes all singular values even if we require only some k values.\\
We have to write code to "sweep" through the matrix, calculate the correct rotation angles for each step, and update two other matrices ($U$ and $V$) at every single step.\\
\textbf{2. Golub-Kahan Algorithm:}\\
Instead of directly computing 
$U,\Sigma,V $, we do it in two major steps:\\
1. Bidiagonalization using Householder transformations\\
We reduce A to a bidiagonal matrix B, such that:
\begin{align*}
    A =PBQ^\top
\end{align*}
where:\\
P and Q are orthogonal,\\
B has nonzero entries only on its main diagonal and superdiagonal.\\
2.Compute the SVD of the bidiagonal matrix B\\
Once A is reduced to B, we use a Golub-Kahan SVD iteration (a stable QR-like method) to diagonalize B to find \\
U,$\Sigma ,V$.\\
uses implicit QR steps on $B^\top B$,\\
converges to the diagonal matrix of singular values $\Sigma$
After Iteration we get:
\begin{align*}
    B=U_B\Sigma V_B^\top\\
    A=(PU_B)\Sigma (QV_B)\\
    A=U\Sigma V\\
    \text{where,} U=PU_B\\
    V=QV_B
\end{align*}
it is extremely complex.\\
Just like Jacobi, it's designed to compute the Full SVD, which is not our goal.\\
\textbf{3. Power Iteration Algorithm:}\\
In this process, we find the largest eigen value and eigen vector of $\vec{A^\top A}$
For Example:\\
let 
\begin{align*}
    \vec{A}=\begin{bmatrix}
        1 &1\\0 &1
    \end{bmatrix}
\end{align*}
The power iteration finds the largest eigenvector of $A^T A$.
\textbf{Calculate $A^T A$:}
    $$
    A^T = \begin{bmatrix} 1 & 0 \\ 1 & 1 \end{bmatrix}
    \quad
    A^T A = \begin{bmatrix} 1 & 0 \\ 1 & 1 \end{bmatrix} \begin{bmatrix} 1 & 1 \\ 0 & 1 \end{bmatrix} = \begin{bmatrix} 1 & 1 \\ 1 & 2 \end{bmatrix}
    $$

 \textbf{Pick a random starting vector $\mathbf{v}_0$:} Let's start with
    $$
    \mathbf{v}_0 = \begin{bmatrix} 1 \\ 1 \end{bmatrix}.
    $$

\textbf{Iterate:} $\mathbf{v}_{\text{new}} = (A^T A) \cdot \mathbf{v}_{\text{old}}$
    \begin{itemize}
        \item \textbf{Iteration 1:} $\mathbf{v}' = \begin{bmatrix} 1 & 1 \\ 1 & 2 \end{bmatrix} \begin{bmatrix} 1 \\ 1 \end{bmatrix} = \begin{bmatrix} 2 \\ 3 \end{bmatrix}$
        \quad Normalize it:
        $$
        \lVert \mathbf{v}' \rVert = \sqrt{2^2 + 3^2} = \sqrt{13} \approx 3.605
        $$
        $$
        \mathbf{v}_1 = \begin{bmatrix} 2/3.605 \\ 3/3.605 \end{bmatrix} = \begin{bmatrix} 0.555 \\ 0.832 \end{bmatrix}
        $$

        \item \textbf{Iteration 2:} $\mathbf{v}' = \begin{bmatrix} 1 & 1 \\ 1 & 2 \end{bmatrix} \begin{bmatrix} 0.555 \\ 0.832 \end{bmatrix} = \begin{bmatrix} 1.387 \\ 2.219 \end{bmatrix}$
        \quad Normalize it:
        $$
        \lVert \mathbf{v}' \rVert = \sqrt{1.387^2 + 2.219^2} \approx 2.617
        $$
        $$
        \mathbf{v}_2 = \begin{bmatrix} 1.387/2.617 \\ 2.219/2.617 \end{bmatrix} = \begin{bmatrix} 0.530 \\ 0.848 \end{bmatrix}
        $$
    \item \textbf{Iteration 3:}
    $$
    \mathbf{v}' = \begin{bmatrix} 1 & 1 \\ 1 & 2 \end{bmatrix} \begin{bmatrix} 0.530 \\ 0.848 \end{bmatrix} = \begin{bmatrix} 1.378 \\ 2.226 \end{bmatrix}
    \quad \text{Normalize it:} \quad \lVert \mathbf{v}' \rVert = \sqrt{1.378^2 + 2.226^2} \approx 2.618
    $$
    $$
    \mathbf{v}_3 = \begin{bmatrix} 1.378/2.618 \\ 2.226/2.618 \end{bmatrix} = \begin{bmatrix} 0.526 \\ 0.850 \end{bmatrix}
    $$
\end{itemize}

The vector is converging! We can stop here. This is our first right singular vector, $\mathbf{v}_1$.
$$
\mathbf{v}_1 = \begin{bmatrix} 0.526 \\ 0.850 \end{bmatrix}
$$

\begin{itemize}
    \item $\mathbf{u}'_1 = A \mathbf{v}_1 = \begin{bmatrix} 1 & 1 \\ 0 & 1 \end{bmatrix} \begin{bmatrix} 0.526 \\ 0.850 \end{bmatrix} = \begin{bmatrix} 1.376 \\ 0.850 \end{bmatrix}$
\\
    \item $\sigma_1 = \lVert \mathbf{u}'_1 \rVert = \sqrt{1.376^2 + 0.850^2} \approx \sqrt{2.616} \approx 1.618$
\\
    \item $\mathbf{u}_1 = \mathbf{u}'_1 / \sigma_1 = \frac{1}{1.618} \begin{bmatrix} 1.376 \\ 0.850 \end{bmatrix} = \begin{bmatrix} 1.376/1.618 \\ 0.850/1.618 \end{bmatrix} = \begin{bmatrix} 0.850 \\ 0.525 \end{bmatrix}$
\end{itemize}

We have found the largest singular triplet:
$(\sigma_1, \mathbf{u}_1, \mathbf{v}_1)$\\\\
Hence I chose this method as we can find the top k eigen values and eigen vectors by applying power iteration and matrix deflation k times. which results in largest k eien values and forms new $A_k$ matrix.
(Matrix deflation: Subtract this triplet's information from the matrix, creating a new, "deflated" matrix.)\\($A_{new} = A - A_1$ \\
$A_1 = \sigma_1 u_1 v_1^\top$)

\section*{\textbf{Error Analysis}}
for error analysis I used Frobenuis norm.\\
For an $m \times n$ matrix $A$, with elements $a_{ij}$ the Frobenius norm is defined as the square root of the sum of the absolute squares of all its elements:
\begin{align*}
\|A\|_F = \sqrt{\sum_{i=1}^{m} \sum_{j=1}^{n} |a_{ij}|^2}
\end{align*}
we calculate the Forbenius norm of the actual image matrix($\|A\|_F$) and the matrices generated for different k values($\|A_k\|_F$) then error is:
\begin{align*}
    error =\|A- A_k\|_F\\
    \text{Relative error}=(\|A - A_k\|_F)/\|A\|_F \\
    \text{Percentage error}= \text{Relative error} *100
\end{align*}
\begin{table}[h!]
    \centering
    \begin{tabular}[12pt]{ |c| c|}
    \hline
 \textbf{k values and Forbenius Norm } & \textbf{error}\\ 
    \hline
     10, 157594.4621& 15060.9479\\
    \hline
     20, 157954.7769& 10636.4985\\
    \hline
    30, 158081.8578& 8542.4176\\
    \hline
    50, 158191.4815&6188.8484\\
    \hline
    \end{tabular}
    \caption{Observations and errors for globe.jpg}
    \label{table 1.4.28}
\end{table}
\begin{table}[h!]
    \centering
    \begin{tabular}[12pt]{ |c| c|}
    \hline
 \textbf{k values and Forbenius Norm } & \textbf{error}\\ 
    \hline
     5, 21293.2517& 4715.3304\\
    \hline
     10, 21564.3793& 3257.9831\\
    \hline
    20, 21705.0692& 2127.6465\\
    \hline
    40,21777.9139& 1165.9213\\
    \hline
    \end{tabular}
    \caption{Observations and errors for einstein.jpg}
    \label{table 1.4.28}
\end{table}
\begin{table}[h!]
    \centering
    \begin{tabular}[12pt]{ |c| c|}
    \hline
 \textbf{k values and Forbenius Norm } & \textbf{error}\\ 
    \hline
     5, 193238.0008& 11156.6352\\
    \hline
    10, 193426.0230& 7195.0763\\
    \hline
    20, 193522.3057& 3809.5483\\
    \hline
    50, 193556.3033& 1163.1619\\
    \hline
    \end{tabular}
    \caption{Observations and errors for greyscale.png}
    \label{table 1.4.28}
\end{table}
\section*{\textbf{Trade-offs and reflections on Implementation choice-Power Iteration algorithm with matrix Deflation}}
\textbf{Pros:}\\
Simple to Implement:\\
The code that I used just uses the for loop containing a matrix-vector multiplication so it's easier to implement.\\
Efficient for this Project:\\
As our goal is to find different matrices with top k eigen values but not Full SVD, power iteration fits very good. It's fast if k is much smaller than the image dimensions than that of calculating whole SVD.\\\\
\textbf{Cons:}\\
Can have numerical stability issues as errors accumulate during deflation the matrix.\\
\section*{Observation}
1. As the value of k increases image quality increases.\\
2. As the value of k decreases image compression increases.\\
\section*{Approximate Images for different values of k}

\begin{figure}[h!]
    \centering
    \includegraphics[width=0.50\columnwidth]{figs/globe.jpg}
    \caption{given Globe Image}
    \label{fig:placeholder}
\end{figure}
\begin{figure}[h!]
    \centering
    \includegraphics[width=0.55\columnwidth]{figs/globe_approx_k10.jpg}
    \caption{k=10}
    \label{fig:placeholder}
\end{figure}
\begin{figure}[h!]
    \centering
    \includegraphics[width=0.55\columnwidth]{figs/globe_approx_k20.jpg}
    \caption{k=20}
    \label{fig:placeholder}
\end{figure}
\begin{figure}[h!]
    \centering
    \includegraphics[width=0.55\columnwidth]{figs/globe_approx_k30.jpg}
    \caption{k=30}
    \label{fig:placeholder}
\end{figure}
\begin{figure}[h!]
    \centering
    \includegraphics[width=0.55\columnwidth]{figs/globe_approx_k50.jpg}
    \caption{k=50}
    \label{fig:placeholder}
\end{figure}

\begin{figure}[h!]
    \centering
    \includegraphics[width=0.5\columnwidth]{figs/einstein.jpg}
    \caption{Given Einstein Image}
    \label{fig:placeholder}
\end{figure}
\begin{figure}[h!]
    \centering
    \includegraphics[width=0.5\columnwidth]{figs/Einstein_approx_k5.jpg}
    \caption{k=5}
    \label{fig:placeholder}
\end{figure}
\begin{figure}[h!]
    \centering
    \includegraphics[width=0.5\columnwidth]{figs/Einstein_approx_k10.jpg}
    \caption{k=10}
    \label{fig:placeholder}
\end{figure}
\begin{figure}[h!]
    \centering
    \includegraphics[width=0.5\columnwidth]{figs/Einstein_approx_k20.jpg}
    \caption{k=20}
    \label{fig:placeholder}
\end{figure}
\begin{figure}[h!]
    \centering
    \includegraphics[width=0.5\columnwidth]{figs/Einstein_approx_k40.jpg}
    \caption{k=40}
    \label{fig:placeholder}
\end{figure}

\begin{figure}[h!]
    \centering
    \includegraphics[width=0.5\columnwidth]{figs/greyscale.png}
    \caption{Given Greyscale image}
    \label{fig:placeholder}
\end{figure}
\begin{figure}[h!]
    \centering
    \includegraphics[width=0.5\columnwidth]{figs/greyscale_approx_k5.jpg}
    \caption{k=5}
    \label{fig:placeholder}
\end{figure}
\begin{figure}[h!]
    \centering
    \includegraphics[width=0.5\columnwidth]{figs/greyscale_approx_k10.jpg}
    \caption{k=10}
    \label{fig:placeholder}
\end{figure}
\begin{figure}[h!]
    \centering
    \includegraphics[width=0.5\columnwidth]{figs/greyscale_approx_k20.jpg}
    \caption{k=20}
    \label{fig:placeholder}
\end{figure}
\begin{figure}[h!]
    \centering
    \includegraphics[width=0.5\columnwidth]{figs/greyscale_approx_k50.jpg}
    \caption{k=50}
    \label{fig:placeholder}
\end{figure}

\end{document}
